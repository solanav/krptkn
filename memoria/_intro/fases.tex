Para el desarrollo del proyecto, se han seguido las etapas estándar del ciclo de vida del software. Estas son las siguientes:

\begin{itemize}
    \item \textbf{Planteamiento del problema}: Se formalizan los requisitos. Durante esta fase también se decide cuáles de estos son necesarios o centrales al proyecto y cuáles son opcionales. Esto se decidirá basandonos tanto en el objetivo del proyecto como en el tiempo de desarrollo y la dificultad que supondría cada uno de los requisitos.
    \item \textbf{Estudio de soluciones existentes}: Investigación de herramientas que solucionan el problema ya sea completa o parcialmente.
    \item \textbf{Diseño de la solución}: Durante esta fase se decide el \dfn{stack} que se va a usar para la herramienta. Se tienen en cuenta los requisitos decididos durante el planteamiento para elegir lenguaje, base de datos, plataforma, etc.
    \item \textbf{Desarrollo}: Creación de la solución propuesta. Esta fase incluye también pruebas y revisiones.
\end{itemize}

A lo largo del proyecto el diseño se ha mantenido muy estable lo que ha aumentado mucho la velocidad de trabajo. Se han evitado reescrituras de grandes bloques de código y ha sido fácil mantener una imagen global del sistema en todo momento.