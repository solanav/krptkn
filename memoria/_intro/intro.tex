Los metadatos son una clase de datos que nos dan información acerca de otros datos. Son secundarios al contenido de un archivo y normalmente su uso está dirigido a otro software y no a humanos. Algunos ejemplos de metadatos con los que estamos más familiarizados pueden ser las miniaturas de las fotos o los \textit{tags} que nos permiten crear filtros de b\'usqueda.

Aunque la mayoría de metadatos son inofensivos en circunstancias normales, cuando existe un adversario que tiene intención de atacar nuestro sistema informático, los metadatos pueden darles una ventaja inusual. Por ejemplo: metadatos en nuestra página web que revelan que nuestro diseñador gráfico utiliza una versión vulnerable de Adobe Photoshop.