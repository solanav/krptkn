La decisión principal con respecto a bases de datos que se toma en el desarrollo de este proyecto es sobre el uso de SQL o NoSQL. Debido a que los metadatos de diferentes formatos tienen estructuras muy distintas se suelen representar como pares de clave valor como $'Size': '480x360'$. Esto encaja muy bien con bases de datos como MongoDB. A pesar de esto, se ha decidido usar PostgreSQL y usar en la columna que contiene los metadatos de cierto archivo el formato JSON. Podríamos haber usado una columna de texto pero el formato nativo JSON de PostgreSQL nos permite verificar que el formato es correcto. También tenemos la ventaja de poder cambiar de JSON a JSONb en cualquier momento. JSONb (json binario) nos daría una mayor velocidad de procesamiento a cambio de perder velocidad de inserción. Y por último, ya que estoy más familiarizado con SQL y el objetivo del proyecto no es aprender sobre NoSQL, usar PostgreSQL es la decisión más segura y que menos problemas va a dar.