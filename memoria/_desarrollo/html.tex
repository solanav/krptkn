El módulo de parseo de HTML utiliza la librería Floki\footnote{\url{https://github.com/philss/floki}} para facilitar la búsqueda los distintos nodos del HTML. Floki ofrece tres parsers distintos:

\begin{itemize}
  \item mochiweb\_html: Escrito en Erlang, sin dependencias externas, lento.
  \item fast\_html: Usa lexbor\footnote{\url{https://github.com/lexbor/lexbor}}, escrito en C, muy rápido.
  \item html5ever: Usa html5ever\footnote{\url{https://github.com/servo/html5ever}}, escrito en Rust, muy rápido también.
\end{itemize}

Se ha optado por el uso de fast\_html por que ha sido el que menos fallos ha dado al probarlo con páginas web complejas. También es importante mencionar que la velocidad no era un problema en ningún caso, al usar muchos hilos paralelos, los tres cumplen de sobra el requisito de velocidad.

Una vez hemos parseado el HTML se buscan los nodos que contienen una propiedad ''href´´ o ''src´´ ya que son los que más a menudo contienen URLs interesantes. Una vez obtenidos estos nodos, se pasa a la limpieza de las URLs.

Esta limpieza es heurística y pasa por varias etapas:

\begin{enumerate}
  \item Se comprueba si la URL pertenece al dominio objetivo. Si estamos analizando ''example.org´´ la URL ''https://code.jquery.com/jquery-3.6.0.js´´ se descarta.
  \item Agregado o eliminado de barras oblicuas al final de la URL. Por ejemplo ''http://example.org/blog´´ se convierte en ''http://example.org/blog/´´ y ''http://example.org/index.html´´ se queda igual. 
  \item HTML permite usar links del estilo ''/index.html´´ cuando en realidad la URL completa sería ''http://example.org/index.html". Otra de las funciones de la heurística se encarga de completar este tipo de URLs.
  \item Por último se filtran los links que usan otros protcolos como FTP, SSH, IPFS, etc. Por ejemplo, descartamos la URL ''mailto:help@example.org".
\end{enumerate}

Una vez se ha hecho esto, se eliminan los duplicados y los que nos quedan se insertan en la cola de URLs. La cola de URLs se ocupa de comprobar si ya hemos visitado el link que le hemos pasado.