Como ya hemos visto, en el núcleo de la aplicación se encuentra la araña, que se encarga de explorar la página web en profundidad. El funcionamiento básico de la araña consiste en extraer una nueva URL de la tabla de URLs (URL Queue en la parte superior de la figura \ref{FIG:ARQUITECTURA}), hacer una petición GET al servidor y mandar el resultado al filtro que se ocupará de redirigir el contenido al módulo de procesamiento adecuado.

Existen dos casos en los que se pueden producir errores (siempre internos y controlados, no causan un cuelgue). En caso de que no exista ninguna URL en la tabla, se producirá un timeout y se reintentará hasta que alguien inserte una nueva URL a esta. También puede ocurrir que el servidor no responda a tiempo a la petición GET. Esto es ignorado y se trata de la misma forma que el problema anterior.

La librería que se usa para las peticiones HTTP es HTTPoison, inspirada por otra librería de Erlang llamada HTTPotion y usando como base Hackney. La decisión de usar HTTPoison se basa en que es fácil de usar, es madura y muy estable, usa SSL por defecto y no nos va a crear un cuello de botella. Aunque algunos usuarios\footnote{\url{https://blog.appsignal.com/2020/07/28/the-state-of-elixir-http-clients.html}}\footnote{\url{http://big-elephants.com/2019-05/gun/}} han notado problemas de escalado para aplicaciones de mucho tráfico en Hackney, he optado por ignorarlos ya que el cuello de botella de Krptkn es la parte de análisis de metadatos.