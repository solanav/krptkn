Aunque en iteraciones iniciales del proyecto se había planteado la posibilidad de usar inteligencia artificial para la clasificación y filtrado de metadatos, se ha optado por una solución menos compleja: regex.

A través de unas reglas de regex muy simples, se extraen correos electrónicos y se comprueba si el texto de los metadatos contiene o no palabras clave que son indicativas de que los metadatos contienen información sensible.

Los datos filtrados no son descartados. Los datos que pueden ser peligrosos se marcan como tal y se insertan en la base de datos para un posterior análisis estadístico y para la creación de los informes pertinentes.