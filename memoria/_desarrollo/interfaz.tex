Una vez la funcionalidad básica del proyecto estaba terminada, se empezó a desarrollar una interfaz web para facilitar el uso por parte de usuarios sin experiencia. También ha resultado extremadamente útil para encontrar problemas en la aplicación.

La decisión de usar un framework web en lugar de una interfaz de terminal o una interfaz de usuario nativa se debe a que Krptkn está pensado para ser ejecutado en un servidor sin interfaz gráfica y para este tipo de despliegues lo más adecuado es una interfaz web a la que puedas acceder desde un ordenador en la misma red local. También se podría configurar para su uso a través de internet aunque el software no esta pensado para aceptar múltiples usuarios a la vez y podría suponer un problema de seguridad.

El framework web que utiliza Krptkn es Phoenix. Este es el estándar para desarrollo de aplicaciones web en Elixir y es también con el que estoy más familiarizado. Aunque en un principio se consideró la idea de usar la variante LiveView de Phoenix que hace uso de websockets para actualizar la interfaz sin necesitar código Javascipt, finalmente se ha decidido usar Phoenix en conjunto con APIs JSON para actualizar los datos de la interfaz. Esto nos da más flexibilidad en el futuro para añadir nuevas funciones y aumenta mucho la velocidad de desarrollo de la parte web (en la cual no quería invertir demasiado tiempo).