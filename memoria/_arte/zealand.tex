La ''Metadata Extraction Tool´´ de la biblioteca nacional de Nueva Zelanda tiene como objetivo la extracción de metadatos de archivos. No tiene funciones de análisis web ni crawling.

El proposito principal de esta herramienta era para uso interno en la biblioteca nacional, extrayendo metadatos de los archivos allí almacenados y preservarlos. Es por ello que no actúa como competidor directo a Krptkn pero sigue siendo importante analizarla para las funciones de extracción, normalización y almacenamiento.

Las características más notorias de esta herramienta son las siguientes:
\begin{itemize}
  \item Soporte tanto para Windows como para Linux.
  \item Alto rendimiento para la extracción. La herramienta analiza solo una pequeña parte de los archivos dados.
  \item Normalización de los metadatos a XML. Todos son convertidos a XML para facilitar el almacenamiento y los procesos de búsqueda posteriores.
  \item Soporte para muchos formatos. El principal atractivo es la cantidad de archivos de oficina que puede analizar: MS Word, Word Perfect, Open Office, MS Works, MS Excel, MS PowerPoint y PDF. Estos son relativamente comunes y al ser para uso interno suelen contener muchos metadatos reveladores.
  \item Cuenta con una interfaz de usuario escrita en Java y puede ser utilizada en forma de librería. Esto es ideal para nuestro caso de uso: una interfaz para visualizar resultados y una librería para un análisis no supervisado. 
\end{itemize}

El principal inconveniente de esta herramienta es su antigüedad. La última versión es de 2014 y muchos de los formatos soportados no han sido probados con nuevas versiones. Además, muchos de estos son para archivos que no se usan en las empresas modernas (Word Perfect por ejemplo).