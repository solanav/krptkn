FOCA o Fingerprinting Organizations with Collected Archives cumple una tarea muy similar a la de Krptkn. Descarga archivos, extrae los metadatos (haciendo uso de una librería interna en lugar de una externa como Krptkn) y los presenta al usuario.

La principal diferencia entre Krptkn y FOCA es que FOCA utiliza los resultados de buscadores web como Google o DuckDuckGo para encontrar los archivos que analiza mientras que Krptkn tiene un \dfn{crawler} interno que se encarga de analizar la web objetivo en profundidad.

FOCA tiene múltiples métodos para analizar un objetivo aparte del uso de motores de búsqueda:

\begin{itemize}
  \item \textbf{Búsqueda a través de DNS}. Se encuentran nuevos dominios e IPs a través de peticiones a servidores DNS. También se hacen uso de registros PTR en caso de que tengamos IPs que queramos escanear.
  \item \textbf{Búsqueda recursiva}. Cada IP nueva que se encuentra, se añade como nuevo objetivo y se analiza. 
  \item \textbf{Fuerza bruta}. Se usan diccionarios para probar nombres comunes contra el servidor DNS.
  \item \textbf{Predicción de DNS}. Se detectan patrones de nombres para probar nuevas queries contra el servidor DNS.
  \item \textbf{Robtex}. Un servicio externo que se usa para encontrar nuevos dominios asociados con el objetivo.
\end{itemize}

A pesar de tener unas características tan atractivas, FOCA solo puede ser ejecutado en Windows y la base de datos que utiliza es SQL Server 2014 por lo que queda relegado a un uso más espontáneo. La herramienta es muy útil para equipos de auditoría de seguridad buscando hacer un análisis en profundidad de una empresa, pero no tanto para un uso continuado en forma de servicio.
