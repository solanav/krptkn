La API, o ``Application Programming Interface'', proporciona una interfaz para la comunicación entre la página web y el núcleo de la aplicación a través de HTTP. Se ha utilizado el framework Phoenix para su desarrollo y las respuestas se dan en formato JSON por su amplica compatibilidad. En la figura \ref{FIG:JSONAPI} se puede observar un ejemplo de respuesta tras unos minutos analizando una web. La petición se realiza a una URL como ``http://localhost:4000/state/last\_metadata''.

\begin{figure}[jsonapi]{FIG:JSONAPI}{Metadatos mostrados en la API}
        \image{\textwidth}{}{json_api}
\end{figure}

La API proporciona acceso a muchos más datos sobre el estado interno de la aplicación. Algunos ejemplos son:

\begin{itemize}
  \item CPU, memoria y procesos de la máquina virtual de Erlang.
  \item Estadísticas sobre el número de metadatos o de URLs.
  \item Últimos metadatos y marcados o no como peligrosos que se han encontrado.
  \item Últimas URLs visitadas.
\end{itemize}

Estos \dfnpl{endpoint} se usan a través de la interfaz como veremos más adelante. La excepción es los de CPU, memoria y procesos. Estos últimos fueron utlizados durante la fase de depurado del proyecto para mejorar el rendimiento y el uso de memoria de Krptkn pero no en su versión final.
