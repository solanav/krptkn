El núcleo de la aplicación consiste en todas las partes explicadas a lo largo del capítulo \ref{CAP:SOLUCION}. A continuación se puede un ejemplo de uso de las funciones de Krptkn (sacado del módulo Krptkn.Application):

\begin{minted}{python}
def manual_start(initial_url) do
    # Creamos un objeto URI con la URL
    initial_uri = URI.parse(initial_url)
    
    # Creamos una lista de URLs que contiene la proporcionada y extras generadas por Krptkn
    initial_urls = [initial_url | Krptkn.Prelaunch.dictionary(initial_uri)]

    # Cada URL la metemos en la cola
    for url <- initial_urls do
        Krptkn.UrlQueue.push(url)
    end
end
\end{minted}

Se puede encontrar la documentación (sin el código fuente) en \url{https://solanav.github.io/krptkn_docs/Krptkn.html}. Esta contiene información sobre los parámetros de todas las funciones del núcleo y ejemplos de uso para las más importantes.