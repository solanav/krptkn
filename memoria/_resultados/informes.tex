Los informes generados constan de una portada (ver figura \ref{FIG:PORTADAINFORME}) que contiene la información más relevante. Está pensada para ser suficiente en caso de no tener tiempo para leer el documento completo.

\begin{figure}[portada]{FIG:PORTADAINFORME}{Portada del informe generado por Krptkn}
        \image{\textwidth}{}{portada}
\end{figure}

Además de el identificador de la sesión, se muestran en la portada el dominio principal, la URL desde la que se empezó a analizar la web, el nivel de peligro estimado y la fecha de inicio y fin del análisis.

El nivel de peligro se calcula dividiendo el número total de metadatos detectados como peligrosos entre el número total de archivos con metadatos que se ha encontrado.

El resto del documento contiene gráficos y tablas que muestran en más detalle cuales han sido los metadatos encontrados, las URLs más visitadas y los tipos de archivo más común en la web objetivo. Se puede ver un ejemplo de estas secciones en la figura \ref{FIG:GRAFICOINFORME}.

\begin{figure}[graficoinforme]{FIG:GRAFICOINFORME}{Ejemplo de un gráfico en el informe generado por Krptkn}
        \image{\textwidth}{}{ejemplo}
\end{figure}
