Los metadatos son ''datos que nos dan información sobre otros datos". Por ejemplo: la fecha de edición de un archivo, el autor de un documento o el programa usado para editar una fotografía. La mayor parte del software que se ejecuta en nuestros ordenadores guarda esta información sobre nuestros archivos sin interacción por parte de los usuarios. Esto no tiene por que suponer un problema ya que la mayoría es inofensiva (fecha de guardado, dimensiones de una foto, licencia de un archivo, etc) pero debemos tener cuidado de eliminar la que no lo es (geolocalización, nombres, comentarios privados, etc).

Aunque la gran mayoría de las redes sociales y servicios de subida de ficheros ya se ocupan de eliminar los metadatos de forma automática, muchas empresas más pequeñas no lo hacen. Y aunque lo intentaran, actualmente no existe una herramienta o librería estándar que facilite esta tarea.

El propósito de este trabajo es el desarrollo de un framework que sirva para extraer estos metadatos de una página web, filtrar los que sean potencialmente peligrosos y generar informes con toda la información relevante. 

\palabrasclave{metadatos, seguridad, framework, tesis, trabajo fin de grado}
