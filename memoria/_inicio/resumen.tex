Los metadatos son "datos que nos dan información sobre otros datos". Por ejemplo: la fecha de edición de un archivo, el autor de un documento o el programa usado para editar una foto.

La mayor parte del software que se ejecuta en nuestros ordenadores guarda mucha información sobre nuestros archivos. Gran parte de esta información es inofensiva (fecha de guardado, dimensiones de una foto, licencia de un archivo, etc) pero debemos tener cuidado de eliminar la que no lo es (geolocalización, nombres, comentarios privados, etc).

La gran mayoría de las redes sociales y servicios de subida de ficheros eliminan los metadatos de forma automática pero muchas empresas más pequeñas no lo hacen. Y aunque lo intentaran corregir, actualmente no existe una herramienta o librería estándar que facilite esta tarea.

El propósito de este trabajo es el desarrollo de un framework que sirva para extraer estos metadatos de una página web, filtrar los que sean potencialmente peligrosos y generar informes con toda la información relevante. 

\palabrasclave{metadatos, seguridad, framework, tesis, trabajo fin de grado}
