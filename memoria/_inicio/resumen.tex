Los metadatos son una parte muy importante en nuestro día a día. A pesar de esto, no los usamos directamente. Solo otros programas se aprovechan de los metadatos de ciertos archivos para facilitar su uso y visualización. Es por esta razón que muchos de los archivos que se suben a internet contienen metadatos que pueden filtrar información sensible. Aunque las gran mayoría de las redes sociales y otros servicios de subida de archivos eliminan los metadatos de forma automática, muchas empresas más pequeñas no lo hacen. Y aunque estas empresas intentaran visualizar estos metadatos, actualmente no existe una herramienta o librería estándar que permita la recopilación automática de metadatos en páginas web de forma fácil.

A lo largo de este trabajo, se desarrollará un framework que sirva para la extracción de estos metadatos sin necesidad de tener conocimientos técnicos. Este framework, denominado "Krptkn", va a ser desarrollado en el lenguaje de programación Elixir con algunos módulos escritos en C para aprovechar algunas librerías interesantes.

\palabrasclave{metadatos, seguridad, framework, tesis, trabajo fin de grado}
