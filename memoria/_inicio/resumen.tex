Los metadatos son una parte muy importante en nuestro d\'ia a d\'ia. A pesar de esto, no los usamos directamente. Solo otros programas se aprovechan de los metadatos de ciertos archivos para facilitar su uso y visualizaci\'on. Es por esta raz\'on que muchos de los archivos que se suben a internet contienen metadatos que pueden filtrar informaci\'on sensible. Aunque las gran mayor\'ia de las redes sociales y otros servicios de subida de archivos eliminan los metadatos de forma autom\'atica, muchas empresas m\'as peque\~nas no lo hacen. Y aunque estas empresas intentaran visualizar estos metadatos, actualmente no existe una herramienta o librer\'ia est\'andar que permita la recopilaci\'on autom\'atica de metadatos en p\'aginas web de forma f\'acil.

A lo largo de este trabajo, se desarrollar\'a un framework que sirva para la extracci\'on de estos metadatos sin necesidad de tener conocimientos t\'ecnicos. Este framework, denominado "Krptkn", va a ser desarrollado en el lenguaje de programaci\'on Elixir con algunos m\'odulos escritos en C para aprovechar algunas librer\'ias interesantes.

\palabrasclave{metadatos, seguridad, framework, tesis, trabajo fin de grado}
