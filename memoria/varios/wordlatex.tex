\section{Ventajas e inconvenientes de \LaTeXe}

El gusto por el LATEX depende de la forma de trabajar de cada uno. La principal virtud
es la facilidad de formatear cualquier texto y la robustez. Incluir referencias a capítulos, secciones, figuras, tablas, etc.\ es
inmediato. Las ecuaciones quedan estupendamente, la escritura se puede realizar modularizada y estructurada. Con estilos como el presentado se reduce considerablemente la utilización de paquetes complejos reduciendo su uso a comandos simples aunque liminatos

\newacronym{wytiwyg}{WYTIWYG}{What You Think Is What You Get}
\newacronym{wysiwyg}{WYSIWYG}{What You See Is What You Get}

El principal inconveniente de \LaTeXe radica en la necesidad de aprender un conjunto de
comandos para generar los elementos que queremos. Cuando se está acostumbrado a un
entorno ``lo que veo es lo que obtengo'' (\acs{wysiwyg}) es difícil cambiar la mentalidad a un entorno del tipo ``lo que pienso es lo que obtengo'' (\acs{wytiwyg}) como \LaTeXe.

Por otro lado, en general será muy complicado cambiar el formato para desviarnos de
la idea original de sus creadores del estilo. No es imposible, pero sí muy difícil. En muchos casos, comno en el tipo de documentos a los que está dirigido este estilo de \LaTeXe es una ventaja y no un inconveniente en el caso de querer obtener una imagen corporativa en los documentos.

\section{Ventajas e inconvenientes de Word\textsuperscript{\textregistered}}

La ventaja mayor del Word\textsuperscript{\textregistered} es que permite configurar el formato muy fácilmente. Para
las ecuaciones tradicionalmente ha proporcionado pésima presentación. Sin embargo, el software adicional
Mathtype\textsuperscript{\textregistered} solventa este problema, incluyendo una apariencia muy profesional y
cuidada. Incluso permite utilizar un estilo similar al de \LaTeXe. Además, aunque el Word\textsuperscript{\textregistered}
incluye sus propios atajos para escribir ecuaciones, Mathtype\textsuperscript{\textregistered} admite también la escritura de ecuaciones utilizando los mismos comandos que \LaTeXe.

Trabajar con títulos, referencias cruzadas e índices es un engorro, por no decir nada sobre
la creación de una tabla de contenidos. Resulta muy frecuente que alguna referencia quede
pérdida o huérfana y aparezca un mensaje en negrita indicando que no se encuentra. Algunos autores hacen todas estas referencias manualmente lo significa que cualquier cambio supone un arduo trabajo rehaciendo las referencias de todo el documento.
Los estilos permiten trabajar bien definiendo la apariencia, pero también puede desembocar
en un descontrolado incremento de los mismos. Además, es muy probable que
Word\textsuperscript{\textregistered} se quede colgado, sobre todo al trabajar con copiar y pegar de otros textos y cuando
se utilizan ficheros de gran extensión, como es el caso de un libro.

\section{¿Cuál elijo?}

\index{eigenvalue}
En cualquier caso las tipografías, colores, distancias entre párrados, interlineados, encabezamientos y estructura del docuemnto debe coincidir con el aquí presentado. Dado que sólo se aporta el estilo de \LaTeXe\ se recomienda su uso aunque no es obligatorio y es decisión del autor elegir.
